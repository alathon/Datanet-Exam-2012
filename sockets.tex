\section{Sockets}
Sockets sit in between a process and the underlying layers - Providing an easy
bridge between the application and transport layers. An internet socket, as is
the topic here, has a local IP:Port, a remote IP:Port, and a Protocol. The protocol
can be UDP, TCP, or one of the other transport-layer protocols.

\begin{itemize}
    \item A UDP socket is a connectionless socket, using UDP as the transport layer mechanism.
    \item A TCP socket is a connection-oriented socket that uses TCP or SCTP.
\end{itemize}

A socket pair is a pair of two communicating sockets (One on the local machine,
one on the remote machine), which together form a 4-tuple (Source IP, destination IP,
source port, destination port). TCP socket 4-tuples are assigned a socket number,
where-as each local unique UDP socket is assigned a socket number.

Raw sockets are also available, and are used in networking equipment for ICMP,
OSPF, IGMP and similar.

\subsection{TCP sockets}
With TCP sockets, a persistent connection must be initiated and maintained. It
is the responsibility of the client to initiate the connection, and so the server
must be listening in the first place. This means:

\begin{itemize}
    \item Server must keep a listen socket open and waiting.
    \item During the 3-way handshake, when the client first contacts the server,
        it immediately allocates a new socket to respond to the client with.
    \item At the end of the 3-way handshake, the client and server talk through
        the servers new socket. This is called the \textbf{connection socket}
\end{itemize}

Since TCP is stateful, connection-oriented and duplex, both sides both send
and receive bytes through their respective sockets. 

Roughly, the process from server can be illustrated as follows:

\begin{enumerate}
    \item Server creates welcome/listen socket.
    \item Server calls welcomeSocket.accept() which blocks/waits for input
    \item When incoming connection request arrives, create connection socket.
    \item Read data from connection socket, and write reply to connection socket.
\end{enumerate}

The process from client can be illustrated as follows:

\begin{enumerate}
    \item Client creates socket and connects to server.
    \item Client sends request using client socket.
    \item Client reads reply from client socket.
\end{enumerate}

Sockets are implemented with streams attached to them. An output stream is for
sending data \textbf{out} to the destination, and an input stream for \textbf{receiving}
data from the destination through the socket. On both sides.

\subsection{UDP sockets}
Unlike TCP sockets, UDP sockets don't need to verify anything about the receiving
socket, do not need to handshake, and so on. This means that you can successfully
send data through UDP to a server that may or may not be listening, which is in
the nature of UDP to begin with.

This has a number of implications:
\begin{enumerate}
    \item The server doesn't need a separate socket for each client.
    \item Data packets sent through a UDP socket must carry their destination
        tuple, always.
    \item There are no streams attached to UDP sockets.
    \item The receiving side must unwrap each packet to discover destination
        address.
\end{enumerate}

The process from server might look like this:
\begin{itemize}
    \item Server creates UDP socket to read incoming requests.
    \item Server reads request from server socket.
    \item Server writes reply to server socket, specifying destination host
        address \& port number.
\end{itemize}

The process from client might look like this:
\begin{itemize}
    \item Client creates UDP socket.
    \item Client creates datagram destined for destination address \& port
        and sends using UDP socket.
    \item Client reads reply from socket.
\end{itemize}

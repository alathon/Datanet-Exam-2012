\section{Packet-switched networks}
Packet switched networks differ from circuit-switched
networks in that they do not guarantee bandwidth by dedicating
a connection to the purpose. 

Packets are data that you want to send through the system, possibly
split up into many smaller packets, to fit within the size constraints
of routers and switches.

\subsection{ISPs}
There are usually 3 tiers of ISPs:
\begin{enumerate}
    \item Tier 1 ISPs, directly connectd to other tier-1 ISPs. International.
        \subitem Known as the Internet backbone. Sprint, Verizon, etc.
    \item Tier 2 ISPs, which are connected to a/a few tier 1 ISPs
        \subitem Typically provides regional coverage.
        \subitem 'Customer' of a tier-1 ISP
        \subitem Has end-point customers ('real' customers)
    \item Tier 3 and below
        \subitem Even more regional usually, and will connect to the tier
        above.
\end{enumerate}

Tier 1 forms the backbone of everything, and tier-2 connect directly to those
and will have customers. Its like leasing out a cable, and a sub-section of it
is then leased on to someone else, who then leases on a subsection of that seciton.

\subsection{Delay / Loss / Throughput}
\subsubsection{Processing delay}
Time required to examine a packet's header, and determine where to direct the
packet. Microseconds or less.

\subsubsection{Queueing delay}
The delay before a packet is transferred into the limited-space queue of the
link. 

\textbf{Traffic intensity} = La/R where: \\
L = packet size\\
a = average arrival rate of packets\\
R = transmission rate\\

If La/R is over 1, then the queue will increase without bound, because you're
getting more incoming traffic than you're pushing out.

\textbf{Packet loss} occurs when the queue is 100\% full, at that point the
packet is dropped.

\subsubsection{Transmission delay}
The delay to push all of the bits of a packet onto the link, measured as L/R
where L is the size in bits and R is the transmission rate in bits/second.

\subsubsection{Propagation delay}
The delay to go from link A to link B. Delay is equal to the distance travelled
divided by the propagation speed of the link. Usually measured in metres. Propagation
speed is usually at or close to the speed of light (Depending on whether its fibre,
twisted-pair copper wire, and so on). Propagation delays are usually in the order
of milliseconds, between links.

\textbf{Nodal delay} = processing + queue + transmission + propagation

The total delay between two nodes A and B, is the sum of nodal delay between each
point the packet has to travel along, to get from A to B. If, for example, it
needs to pass through 5 routers - Then it is the sum of the nodal delay between
those 5 routers.

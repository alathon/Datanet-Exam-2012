\section{TCP}
\subsection{Flow control}
TCP uses an end-to-end flow control protocol to avoid having the sender send
data too fast for the TCP receiver to receive and process it reliably. Having
a mechanism for flow control is essential in an environment where machines of
diverse network speeds communicate. For example, if a PC sends data to a
smartphone that is slowly processing received data, the smartphone must regulate
the data flow so as not to be overwhelmed. TCP uses a sliding window flow
control protocol. In each TCP segment, the receiver specifies in the receive
window field the amount of additionally received data (in bytes) that it is
willing to buffer for the connection. The sending host can send only up to
that amount of data before it must wait for an acknowledgment and window update
from the receiving host.

TCP sequence numbers and receive windows behave very much like a clock. The
receive window shifts each time the receiver receives and acknowledges a new
segment of data. Once it runs out of sequence numbers, the sequence number
loops back to 0. When a receiver advertises a window size of 0, the sender stops
sending data and starts the persist timer. The persist timer is used to protect
TCP from a deadlock situation that could arise if a subsequent window size
update from the receiver is lost, and the sender cannot send more data until
receiving a new window size update from the receiver. When the persist timer
expires, the TCP sender attempts recovery by sending a small packet so that
the receiver responds by sending another acknowledgement containing the new
window size.

\subsection{Congestion avoidance}
To avoid congestion collapse, TCP uses a multi-faceted congestion control strategy. 
For each connection, TCP maintains a congestion window, limiting the total number 
of unacknowledged packets that may be in transit end-to-end. This is somewhat 
analogous to TCP's sliding window used for flow control. TCP uses a mechanism 
called slow start to increase the congestion window after a connection is 
initialized and after a timeout. It starts with a window of two times the maximum 
segment size (MSS). Although the initial rate is low, the rate of increase is 
very rapid: for every packet acknowledged, the congestion window increases by 1
 MSS so that the congestion window effectively doubles for every round trip time
 (RTT). When the congestion window exceeds a threshold ssthresh the algorithm
 enters a new state, called congestion avoidance. In some implementations (e.g.,
 Linux), the initial ssthresh is large, and so the first slow start usually ends
 after a loss. However, ssthresh is updated at the end of each slow start, and
 will often affect subsequent slow starts triggered by timeouts.

\subsubsection{Congestion avoidance in general:} As long as non-duplicate ACKs
are received, the congestion window is additively increased by one MSS every
round trip time. When a packet is lost, the likelihood of duplicate ACKs being
received is very high (it's possible though unlikely that the stream just
underwent extreme packet reordering, which would also prompt duplicate ACKs).
The behavior of Tahoe and Reno differ in how they detect and react to packet loss:

\subsubsection{Tahoe}
Triple duplicate ACKS are treated the same as a timeout. Tahoe will perform
"fast retransmit", reduce congestion window to 1 MSS, and reset to slow-start
state.

\subsubsection{Reno}
If three duplicate ACKs are received (i.e., four ACKs acknowledging the
same packet, which are not piggybacked on data, and do not change the receiver's
advertised window), Reno will halve the congestion window, perform a fast
retransmit, and enter a phase called Fast Recovery. If an ACK times out, slow
start is used as it is with Tahoe.

\textbf{Fast Recovery}. (Reno Only) In this state, TCP retransmits the missing packet
that was signaled by three duplicate ACKs, and waits for an acknowledgment of
the entire transmit window before returning to congestion avoidance. If there
is no acknowledgment, TCP Reno experiences a timeout and enters the slow-start
state.

Both algorithms reduce congestion window to 1 MSS on a timeout event.

\subsubsection{Slow Start}
TODO

\subsubsection{Fast retransmit}
TODO

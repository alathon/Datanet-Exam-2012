\section{NAT}
NAT (Network Address Translation) is used by routers to control the flow of
packets between the WAN (Wide Area Network) and the LAN. When a user is sitting
behind a router, the router appears to be a communicating client, when the real
client is sitting behind the router. The router maps WAN addresses to LAN
addresses. When a user behind a router wants to communicate with the outside
world, he opens a specific port to communicate over. The outside world only
knows about this port, and sends data to the router through this port. The
router holds a "NAT translation table", which maps open communication from the
WAN to the LAN, so it knows where the incoming data is supposed to go.

\subsection{UPnP}
If an application wants to work around NAT, and the applications end-point of
communication is NAT UPnP compatible (the clients network needs to be
compatible too), the application may create a public IP and a public port, which
appears to be the clients actual IP and port. Other contacts through the
application establish connection through this public mapping, and the router
makes sure it is sent to the correct client.
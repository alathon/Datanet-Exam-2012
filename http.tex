\section{HTTP}
\section{Cookies}
Since HTTP is stateless, there is a need to convey some measure of state in
a different manner. For example, 
\section{Web caches}
An HTTP web cache is a proxy that sits in front of a web-server, intercepting
calls in order to potentially return to the caller a cached version of an object.
Most caches will store a key-value map between URL's and the response from the web-server
as an 'object', and return that.

HTTP headers such as If-Modified-Since allow for the effecient retrieval of pages,
by only re-querying under the conditions specified. Caching is very important for
maintaining the speed of responses, especially in situations where website(s) have
a large number of concurrent users. Most web caches will also keep the cache entirely
in-memory, making response time in the order of milliseconds.

\section{REST}
REST is a Web service design model. Something that conforms to the REST model is
generally called a \textbf{RESTful} service. HTTP is RESTful.

Something is RESTful when it caches, is stateless, is separated into client and server,
provides a uniform interface, obides by a layered system (Cannot tell if the response
came from a proxy or the web server itself) and can extend the functionality of a
client by transferring executable code.

\section{DHCP}
DHCP is used when a computer wants to connect to a network. It is used
to assign IP addresses to connecting computers.

The computer that wants to connect, sends a DHCP packet to address
0.0.0.0 and the broadcast address 255.255.255.255. The DHCP service
should then give a lease on an address to the connecting
machine.

\begin{enumerate}
    \item Client sends DHCP DISCOVER to 0.0.0.0/32
    \item DHCP server sends DHCP OFFER
    \item Client requests lease with DHCP REQUEST
    \item Server acknowledges lease with DHCP ACK
\end{enumerate}

It is relevant to note that multiple offers can be received, hence the structure
of the above.

\subsection{CIDR}
CIDR is used to calculate the IP address. The format is x.x.x.x/n
where n ranges from 1 to 32, and is called the \textbf{prefix}.

To convert from full decimal notation to CIDR notation, we convert the
decimal notation to binary, and note how many bits are set. This
defines what the prefix should be.

\subsubsection{Calculating number of subnets}
The subnet mask and Class determines how many subnets and hosts you get.

\[Hosts = 2^(HostBits)-2
    Subnets = 2^(SubnetBits)-2
\]

\subsubsection{Calculating subnet address}
Subnet address is obtained by doing a binary AND between IP address and Subnet Mask.

For example:

\begin{align*}
    IP:          &150.10.10.10\\
    Subnet mask: &255.255.252.0 (150.10.10.10/22)\\
    IP Binary:   &10010110.00001010.00001010.00001010\\
    Mask binary: &11111111.11111111.11111100.00000000\\
    Binary AND:  &10010110.00001010.00001000.0000000000 (150.10.8.0)
\end{align*}

\subsubsection{Calculating broadcast address}
Broadcast address is obtained by doing a binary OR between Subnet Address/Network Address
and inverted Subnet Mask

For example:

\begin{align*}
    IP:           &150.10.10.10\\
    Subnet mask:  &255.255.252.0 (150.10.10.10/22)\\
    Subnet addr:  &10010110.00001010.00001000.00000000 (150.10.8.0)
    Inv sub mask: &00000000.00000000.00000011.11111111 (0.0.3.255)
    Binary OR:    &10010110.00001010.00001011.11111111 (150.10.11.255)
\end{align*}

In the same folder as these notes, there are two documents: cidr-subnet-cheatsheet
and subnets. Both of these documents contain tables with common bit fields to
look up the address fields.

\subsection{Forwarding}

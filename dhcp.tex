\section{DHCP}
DHCP is used when a computer wants to connect to a network. It is used
to assign IP addresses to connecting computers.

The computer that wants to connect, sends a DHCP packet to address
0.0.0.0 and the broadcast address 255.255.255.255. The DHCP service
should then give a lease on an address to the connecting
machine. The machine sends a DHCP discover packet, the server answers
with a DHCP offer packet, the machine sends a DHCP request packet, and
finally the server sends a DHCP ACK.

\subsection{CIDR}
CIDR is used to calculate the IP address. The format is x.x.x.x/n
where n ranges from 1 to 32.

To convert from full decimal notation to CIDR notation, we convert the
decimal notation to binary, and note how many bits are set. This
defines what $n$ in the previous example should be.\\
For instance with a subnet mask of 255.255.255.0 would be
192.168.1.1/24 in CIDR notation, since 24 bits are set in the
decimal format. In this case, the network address is 192.168.1.0 and the
broadcast address is 192.168.1.127.

The other way around, we would have 

\subsection{Forwarding}